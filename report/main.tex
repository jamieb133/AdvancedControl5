
%------------DOC TYPE-------------------%
\documentclass[11pt]{article}

%-----------INCLUDES----------------------%
\usepackage[english]{babel}
\usepackage[utf8x]{inputenc}
\usepackage[left=2.5cm, right=2cm, top=2.5cm]{geometry} %margins
\usepackage{amsmath}
\usepackage{graphicx}
\usepackage[colorinlistoftodos]{todonotes}
\usepackage{subfiles}
\usepackage{titlesec}
\usepackage{fancyhdr}
\usepackage{verbatim}
\usepackage[bottom]{footmisc}
\usepackage{float} %CONCRETE FIGURE PLACEMENT
\usepackage{listings} %CODE LISINGS
\usepackage{color}
\usepackage{tikzit}
\usepackage{bm}
\usepackage{tabu}
\usepackage{hhline}
\usepackage[T1]{fontenc}
\usepackage[utf8x]{inputenc}
\usepackage[english]{babel}
\usepackage{csvsimple}
\usepackage{hyperref}
\usepackage{epsfig}
\usepackage{makecell}
\usepackage{array}
\usepackage{xcolor,colortbl}
%\usepackage{pgflibraryarrows}

%----------------TIKZ STYLES---------------%
%\input{3-CommsIP/figs/ptpStyle.tikzstyles}

\setcounter{secnumdepth}{3}

%---------------PAGE STYLE/HEADER-FOOTER-------------------%
\pagestyle{fancy}% Change page style to fancy
\fancyhf{}% Clear header/footer
\fancyhead[R]{2127147b}
%\fancyhead[L]{}
\fancyfoot[R]{\thepage}% \fancyfoot[R]{\thepage}
\renewcommand{\headrulewidth}{0.5pt}
\renewcommand{\footrulewidth}{0.5pt}
\fancypagestyle{plain}{} %put HF on chapter page
 
 % footnote in footer
\renewcommand*\footnoterule{}
 
%-------------TITLE CONFIG--------%
%\titleformat{\chapter}[display]
 % {\normalfont\bfseries}{}{0pt}{\LARGE}

 %-----------FONT--------------%
 %\renewcommand{\familydefault}{ptm}

%-----------GRAPHICS---------------%
\graphicspath{{../figures/}}
\input{control5.tikzstyles}

%-------TABLES----------%
\newcolumntype{?}{!{\vrule width 1pt}}

\definecolor{jamie}{rgb}{1,1,0.82}
\newcommand{\done}{\cellcolor{jamie}done}  %{0.9}
\newcommand{\hcyan}[1]{{\color{teal} #1}}

%-------LISTINGS----------%
\definecolor{codegreen}{rgb}{0,0.6,0}
\definecolor{codegray}{rgb}{0.5,0.5,0.5}
\definecolor{codepurple}{rgb}{0.58,0,0.82}
\definecolor{backcolour}{rgb}{0.95,0.95,0.92}
 
\lstdefinestyle{mystyle}
{
    backgroundcolor=\color{backcolour},   
    commentstyle=\color{codegreen},
    keywordstyle=\color{blue},
    numberstyle=\tiny\color{codegray},
    stringstyle=\color{codepurple},
    basicstyle=\ttfamily\footnotesize,
    breakatwhitespace=false,         
    breaklines=true,                 
    captionpos=b,                    
    keepspaces=true,                 
    numbers=left,                    
    numbersep=5pt,                  
    showspaces=false,                
    showstringspaces=false,
    showtabs=false,                  
    tabsize=1
}


\lstset
{
    emph={BYTE_SWAP},
    emphstyle={\color{orange}},
    otherkeywords = {memset, memcpy},
    style=mystyle
}

\providecommand{\main}{.}

%------------STRUCTURE------------%
\begin{document}

\title{Advanced Control 5 (ENG5009) \\ \large{Lab Assignment}}
\author{2127147b}
\maketitle

\begin{abstract}
    \addcontentsline{toc}{chapter}{Abstract}
    The following report outlines the development and testing of a waypoint following and obstacle avoidance system for the simulation of an autonomous robot in MATLAB.
    The system presented uses fuzzy logic controllers to generate desired turning commands and motor gains, a range of different input types were used alongside basic signal processing techniques to provide the fuzzy controllers with sufficient insight into the surrounding environment.
    The controller was found to produce successful results with the robot travelling to a specific coordinate with a 0.05m radius tolerance. 
    Further development and fine-tuning was carried out to optimise the controller performance for a set of different scenarios.
    All code can be found on GitHub at \cite{github}, relevant code is included in the appendices.  
    \end{abstract}

\section{Introduction}

\section{Methodology}
\subsection{Overview of System}

\begin{figure}[H]
    \centering
    \ctikzfig{./figures/SystemDiagram}
    \caption{Block Diagram of Control System}
    \label{fig:system}
\end{figure}


\pagebreak
\subsection{Task 1: Waypoint Following}
\subsubsection{Overview}
\subsubsection{Fuzzy Sets}
%INPUT SETS
\begin{figure}[H]
    \centering
\includegraphics[scale=0.75]{./figures/HeadingControlInput2.eps}
\caption{Membership Functions for Heading Angle Input}
\end{figure}

%OUTPUT SETS
\begin{figure}[H]
    \centering
\includegraphics[scale=0.75]{./figures/HeadingControlOutput1.eps}
\caption{Membership Functions for Turn Command Output}
\end{figure}

\subsubsection{Rules}
\begin{table}[H]
    \centering  
    \caption{Fuzzy Logic Rules for Path Controller}
    \begin{tabu} to 0.8\textwidth { ? l | l ? l ? l ?}
        \Xhline{2\arrayrulewidth}
        $\psi_{ref}$   & $\psi$ &  \cellcolor{jamie} \textbf{TURN CMD} \\
        %\hhline{|=|=||=|}
        \Xhline{2\arrayrulewidth}
        $S_{-ve}$  &  $S_{-ve}$ &\cellcolor{jamie} $FWD$\\
        \hline
        $S_{-ve}$ &  $SW$ &\cellcolor{jamie} $L_{soft}$\\ 
        \hline
        $S_{-ve}$ &  $W$ &\cellcolor{jamie} $L_{hard}$\\
        \hline
        $S_{-ve}$ & $NW$ &\cellcolor{jamie} $L_{rot}$\\
        \hline
        $S_{-ve}$ &  $N$ &\cellcolor{jamie} $L_{rev}$\\
        \hline
        $S_{-ve}$ & $NE$ & \cellcolor{jamie} $R_{rot}$\\
        \hline
        $S_{-ve}$ & $E$ & \cellcolor{jamie} $R_{hard}$\\
        \hline 
        $S_{-ve}$ & $SE$ & \cellcolor{jamie} $R_{soft}$\\
        \hline
        $S_{-ve}$  & $S_{+ve}$ & \cellcolor{jamie} $FWD$\\
        \Xhline{2\arrayrulewidth}
    \end{tabu}
    
    \label{table:IPfuncs}
    \end{table}
\subsubsection{Verification}

\pagebreak
\subsection{Task 2: Obstacle Avoidance}

\subsubsection{Overview}
\subsubsection{Fuzzy Sets}
%INPUTS
%turn command input the same as output from other controller
\begin{figure}[H]
    \centering
\includegraphics[scale=0.75]{./figures/MotorControlInput2.eps}
\caption{Membership Functions for Radius Input}
\end{figure}

\begin{figure}[H]
    \centering
\includegraphics[scale=0.75]{./figures/MotorControlInput3.eps}
\caption{Membership Functions for Wall Angle Input}
\end{figure}

\begin{figure}[H]
    \centering
\includegraphics[scale=0.75]{./figures/MotorControlInput4.eps}
\caption{Membership Functions for Wall Proximity Input}
\end{figure}

%filtered proximity functions the same as unfiltered

%OUTPUTS
\begin{figure}[H]
    \centering
\includegraphics[scale=0.75]{./figures/MotorControlOutput1.eps}
\caption{Membership Functions for Motor Gain Outputs}
\end{figure}

\begin{figure}
    \centering
\includegraphics[scale=0.75]{./figures/MotorControlOutput3.eps}
\caption{Membership Functions for Filter Cutoff Frequency Output}
\end{figure}

\subsubsection{Rules}
\subsubsection{Verification}

\pagebreak

\section{Results and Testing}
\subsubsection{Assigned Waypoint}
\subsubsection{Wall Tracking}
%average filter
\subsubsection{Perpendicular Approach}
%jolting, reducing flat member func
%filter

\pagebreak
\section{Discussion}
\subsection{Evaluation}
\subsection{Further Work}
%unused variables (velocity, rotational velocity)

\pagebreak
\bibliographystyle{unsrt}
\bibliography{refs}

\pagebreak
\appendix
\section{Main Simulation Code}
\lstinputlisting[language = matlab]{../run_model.m}
\pagebreak
\section{FIRFilter Class}
\lstinputlisting[language = matlab]{../FIRFilter.m}
\section{Lab 1 Answer Sheet}


\end{document}
